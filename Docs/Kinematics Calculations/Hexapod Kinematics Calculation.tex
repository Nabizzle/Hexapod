\documentclass[12pt, letterpaper]{article}

\usepackage[utf8]{inputenc}
\usepackage{graphicx}
\usepackage{parskip}
\usepackage{amsmath}
\usepackage{amssymb}
\usepackage[section]{placeins}
\usepackage{import}
\usepackage{xifthen}
\usepackage{pdfpages}
\usepackage{transparent}
\usepackage{gensymb}

\graphicspath{{../media/}}

\title{Kinematics Calculations for Hexapod Robot}
\author{Nabeel Chowdhury}
\date{\today}

\newcommand{\incfig}[1]{%
    \def\svgwidth{\columnwidth}
    \import{../media/}{#1.pdf_tex}
}

\begin{document}
	\maketitle
	\newpage
	
	\section{Servo Angles for each Leg}
		Looking from above, the first angle of each leg that makes contact with the floor at $P(x,y,z)$ is found as $\theta_C = \tan^{-1} \left( \frac{L_y}{L_x} \right) = \tan^{-1} \left( \frac{P_y}{P_x} \right)$.
		
		\begin{figure}[ht]
			\centering
			\resizebox{0.99\textwidth}{!}{\incfig{Leg Top}}
			\caption{The top view of the leg. $\theta_C$ is the Coax angle of the servo attached to the body of the Hexapod. $L_x$ is the length of the leg from the body in the x direction and $L_y$ is the length of the leg in the y direction.}
		\end{figure}
		
		\newpage
		Looking from the side of the leg, we can define several parameters.
		
		\begin{figure}[ht]
			\centering
			\resizebox{0.88\textwidth}{!}{\incfig{Leg Side}}
			\caption{The side view of the leg. $\theta_F$ is the Femur angle and $\theta_T$ is the tibia angle. $L_C$ is the offset of the femur servo from the body, or the coax length. $L_F$ is the length of the Femur and $L_T$ is the length of the Tibia. $L_x$ is the x distance of the leg's point of contact with the ground. $H$ is the height of the body off of the ground. The rest of the parameters are needed in interim steps to find $\theta_F$ and $\theta_T$.}
		\end{figure}
		
		First, we need to find the length of $A$.
		\begin{flalign*}
			A &= \sqrt{H^2 + \left( L_x - L_C \right)^2} = \sqrt{P^2_z + \left( P_x - L_C \right)^2} &\\
		\end{flalign*}
		
		Using $A$ we can use the law of cosines to find $\theta_F$.
		\begin{flalign*}
			\theta_F &= \theta_\beta - \theta_\alpha &\\			
			\theta_\alpha &= \tan^{-1} \left( \frac{H}{L_x - L_C} \right) &\\
			L_T^2 &= L_F^2 + A^2 -2L_FA \cos (\theta_\beta) &\\ 
			\therefore \theta_F &= \cos^{-1} \left( \frac{L_T^2 - L_F^2 - A^2}{-2L_FA} \right) - \tan^{-1} \left( \frac{H}{P_x - L_C} \right) &\\
		\end{flalign*}
		
		We can also use the law of cosines to find the tibia angle as well.
		\begin{flalign*}
			-\theta_T &= 90\degree - \theta_\gamma &\\			
			A^2 &= L_F^2 + L_T^2 -2L_FL_T \cos (\theta_\gamma) &\\ 
			\therefore \theta_T &= \cos^{-1} \left( \frac{A^2 - L_F^2 - L_T^2}{-2L_FL_T} \right) - 90\degree &\\
		\end{flalign*}
		
		In summary:
		\begin{flalign*}
			& \boxed{\theta_C = \tan^{-1} \left( \frac{P_y}{P_x} \right)} &\\
			& \boxed{\theta_F = \cos^{-1} \left( \frac{L_T^2 - L_F^2 - P^2_z - \left( P_x - L_C \right)^2}{-2L_F\left( \sqrt{P^2_z + \left( P_x - L_C \right)^2} \right)} \right) - \tan^{-1} \left( \frac{-P_z}{P_x - L_C} \right)} &\\
			& \boxed{\theta_T = \cos^{-1} \left( \frac{P^2_z + \left( P_x - L_C \right)^2 - L_F^2 - L_T^2}{-2L_FL_T} \right) - 90\degree } &\\
		\end{flalign*}

	\newpage	
	\section{Converting Leg Origin to Body Origin}
		In order to find how all of the legs will move together, the origins of each leg need to be shifted from their body attachment points to the body center. Shown in the image below is a top down view of the hexapod body with the leg attachment points labeled as $x_n, y_n, z_n$ with $n = 0, 1, 2, 3, 4, 5$ to correspond to the leg numbers in the code. in the default position, the offset of the legs from the body are a known number, but each offset will change as the body rotates in place. This will be explored later.
		\begin{figure}[ht]
			\centering
			\resizebox{0.95\textwidth}{!}{\incfig{Base Hexapod Position}}
			\caption{The top view of the whole body with each leg attachment point labeled with x, y, and x values.}
		\end{figure}
		
		The position of each leg endpoint can be found using the body offset plus the leg's position from the body attachment.
		\begin{flalign*}
			x_n &= B_{xn} + P_{xn} \text{ where: } n = 0,1,2,3,4,5&\\
			y_n &= B_{yn} + P_{yn} \text{ where: } n = 0,1,2,3,4,5&\\
			z_n &= B_{zn} + P_{zn} \text{ where: } n = 0,1,2,3,4,5&\\
		\end{flalign*}		
		
		\subsection{Body Offset Changes Due to Rotation}
			Each body offset will change based on the rotation of the body. This could be solved using trigonometry like for the legs, but a simpler, global solution is to use rotation matrices and find where each leg attachment point is moved to in space. A change in pitch is a rotation around the x axis, a change in roll is a rotation around the y axis, and a change in yaw is a rotation around the z axis.
			\subsubsection{Rotation Matrices}
				The rotation matrices are below:
				
				\textbf{Rotation About the X Axis}
				\begin{equation*}
				R_x = 
					\begin{bmatrix}
						&1 &0 &0 \\
    					&0 &\cos(\theta) &\sin(\theta) \\
    					&0 &-\sin(\theta) &\cos(\theta)
					\end{bmatrix}
				\end{equation*}
				
				\textbf{Rotation About the Y Axis}
				\begin{equation*}
				R_y = 
					\begin{bmatrix}
						&\cos(\theta) &0 &\sin(\theta) \\
    					&0 &1 &0 \\ 
    					&-\sin(\theta) &0 &\cos(\theta)
					\end{bmatrix}
				\end{equation*}
				
				\textbf{Rotation About the Z Axis}
				\begin{equation*}
				R_z = 
					\begin{bmatrix}
						&\cos(\theta) &\sin(\theta)&0 \\
    					&-\sin(\theta) &\cos(\theta) &0 \\
    					&0 &0 &1
					\end{bmatrix}
				\end{equation*}
				
				To find the new endpoints, we do the following calculation:
				\begin{equation*}
					\begin{bmatrix}
						&B_{xn}' \\
    					&B_{yn}' \\
    					&B_{zn}'
					\end{bmatrix}
					= (R_zR_yR_x)^{-1}
					\begin{bmatrix}
						&B_{xn} \\
    					&B_{yn} \\
    					&B_{zn}
					\end{bmatrix}
					\text{ where: } n = 0,1,2,3,4,5
				\end{equation*}
				
				Therefore, the endpoints of each leg in relation to the body center is:
				\begin{equation*}				
					\begin{bmatrix}
						&x_n \\
    					&y_n \\
    					&z_n
					\end{bmatrix}
					= (R_zR_yR_x)^{-1}
					\begin{bmatrix}
						&B_{xn} \\
    					&B_{yn} \\
    					&B_{zn}
					\end{bmatrix}
					+
					\begin{bmatrix}
						&P_{xn} \\
    					&P_{yn} \\
    					&P_{zn}
					\end{bmatrix}
					\text{ where: } n = 0,1,2,3,4,5
				\end{equation*}
			\subsection{Body Offset Changes Due to Body Translation}
				When the body of the hexapod translates in place, the only change that is needed is to add the translation to each offset. Assuming there is a desired translation of $[T_x, T_y, T_z]$, the body offsets will be:
				\begin{equation*}
					\begin{bmatrix}
						&B_{xn}' \\
    					&B_{yn}' \\
    					&B_{zn}'
					\end{bmatrix}
					=
					\begin{bmatrix}
						&B_{xn} \\
    					&B_{yn} \\
    					&B_{zn}
					\end{bmatrix}
					+
					\begin{bmatrix}
						&T_{x} \\
    					&T_{y} \\
    					&T_{z}
					\end{bmatrix}
					\text{ where: } n = 0,1,2,3,4,5
				\end{equation*}
		
			\subsection{Combined Body Offset Changes}
				The combined changes to the body offsets is below:		
				\begin{equation*}
					\begin{bmatrix}
						&B_{xn}' \\
    					&B_{yn}' \\
    					&B_{zn}'
					\end{bmatrix}
					= (R_zR_yR_x)^{-1}
					\begin{bmatrix}
						&B_{xn} \\
    					&B_{yn} \\
    					&B_{zn}
					\end{bmatrix}
					+
					\begin{bmatrix}
						&T_{x} \\
    					&T_{y} \\
    					&T_{z}
					\end{bmatrix}
					\text{ where: } n = 0,1,2,3,4,5
				\end{equation*}
		\section{Hexapod Turning}
			Turning the hexapod is slightly different from the rotation of the body alone as the act of turning the whole hexapod requires rotating the origin point. Here we take leg 1 as an example with a small left turn with an angle of $\alpha$.
		
		\begin{figure}[ht]
			\centering
			\resizebox{0.85\textwidth}{!}{\incfig{Hexapod Turning}}
			\caption{The hexapod taking a small left turn.}
		\end{figure}
		
		\begin{figure}[ht]
			\centering
			\resizebox{0.85\textwidth}{!}{\incfig{Hexapod Turning Zoomed}}
			\caption{Zoomed in view of leg 1 during the left turn}
		\end{figure}
		
		The first thing we need to find is the angle the leg end point makes withe the center of the body before the turn. this is found as:
		\begin{flalign*}
			\theta_a = \tan^{-1} \left(\frac{y_n}{x_n} \right)
		\end{flalign*}				
		
		Now we can use the previous rotation matrices to find the length of the leg from the body.
		\begin{flalign*}
			\theta_a = \tan^{-1} \left(\frac{y_n}{x_n} \right)
		\end{flalign*}
		
		\begin{flalign*}
			\begin{bmatrix}
				&P_{xn} \\
    			&P_{yn} \\
    			&P_{zn}
			\end{bmatrix}
			=
			\begin{bmatrix}
				&x_n \\
    			&y_n \\
    			&z_n
			\end{bmatrix}
			- (R_zR_yR_x)^{-1}
			\begin{bmatrix}
				&B_{xn} \\
    			&B_{yn} \\
    			&B_{zn}
			\end{bmatrix}
			-
			\begin{bmatrix}
				&T_{x} \\
    			&T_{y} \\
    			&T_{z}
			\end{bmatrix}
			\text{ where: } n = 0,...,5
		\end{flalign*}
		
		The length of the leg from the body is $L = \sqrt{P_{xn}^2 + P_{yn}^2}$
		
		In the rotated coordinate frame, we can use the original body offset distances and the length of the leg to find the new leg end point position in the rotated global coordinate frame.
		
		\begin{flalign*}
			P'_{xn} &= L \cos(\theta_a - \alpha) - B_{xn} &\\
			P'_{yn} &= L \sin(\theta_a - \alpha) - B_{yn} &\\
			P'_{zn} &= P_{zn}
		\end{flalign*}
		
		These new leg positions are where they should move to turn the whole body to a new angle and then the legs can left up and move to their default position to keep the robot rotated.
		
		\newpage
		\section{Hexapod Translation in the XY Plane}
		Having the hexapod move in any direction level to the ground is simple. The position of the end points of the legs is the opposite of movement of the center of the body. Therefore, the new position of the legs is found using the following 
		\begin{figure}[ht]
			\centering
			\resizebox{0.95\textwidth}{!}{\incfig{Hexapod Walking}}
			\caption{The top view body when making a motion in the xy plane.}
		\end{figure}
		
		\begin{flalign*}
			P'_{xn} &= P_{xn} - W_x &\\
			P'_{yn} &= P_{yn} - W_y &\\
			P'_{zn} &= P_{zn}
		\end{flalign*}
		
		\newpage
		\section{Hexapod Alternative Walk Cycle}
		When the hexapod makes a motion two alternative sets of legs move. One set stays on the ground to form a tripod and the other moves off the ground and into position. The legs will always for an equilateral triangle. For the legs that move off of the ground, the calculation should be made so that it assumes that the origin moved in the opposite direction of the actual motion. Then the second set of legs comes off the ground and does the same.
		\begin{figure}[ht]
			\centering
			\resizebox{0.95\textwidth}{!}{\incfig{Hexapod Forward Step}}
			\caption{The alternative tripod walk of the hexapod.}
		\end{figure}
		
\end{document}