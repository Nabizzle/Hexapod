\documentclass[12pt, letterpaper]{article}

\usepackage[utf8]{inputenc}
\usepackage{graphicx}
\usepackage{parskip}
\usepackage{amsmath}
\usepackage{amssymb}
\usepackage[section]{placeins}
\usepackage{import}
\usepackage{xifthen}
\usepackage{pdfpages}
\usepackage{transparent}
\usepackage{gensymb}

\graphicspath{{../media/}}

\title{Kinematics Calculations for Hexapod Robot}
\author{Nabeel Chowdhury}
\date{\today}

\newcommand{\incfig}[1]{%
    \def\svgwidth{\columnwidth}
    \import{../media/}{#1.pdf_tex}
}

\begin{document}
	\maketitle
	\newpage
	
	\section{Servo Angles for each Leg}
		Looking from above, the first angle of each leg that makes contact with the floor at $P(x,y,z)$ is found as $\theta_C = \tan^{-1} \left( \frac{L_y}{L_x} \right) = \tan^{-1} \left( \frac{P_y}{P_x} \right)$.
		
		\begin{figure}[ht]
			\centering
			\resizebox{0.99\textwidth}{!}{\incfig{Leg Top}}
			\caption{The top view of the leg. $\theta_C$ is the Coax angle of the servo attached to the body of the Hexapod. $L_x$ is the length of the leg from the body in the x direction and $L_y$ is the length of the leg in the y direction.}
		\end{figure}
		
		\newpage
		Looking from the side of the leg, we can define several parameters.
		
		\begin{figure}[ht]
			\centering
			\resizebox{0.88\textwidth}{!}{\incfig{Leg Side}}
			\caption{The side view of the leg. $\theta_F$ is the Femur angle and $\theta_T$ is the tibia angle. $L_C$ is the offset of the femur servo from the body, or the coax length. $L_F$ is the length of the Femur and $L_T$ is the length of the Tibia. $L_x$ is the x distance of the leg's point of contact with the ground. $H$ is the height of the body off of the ground. The rest of the parameters are needed in interim steps to find $\theta_F$ and $\theta_T$.}
		\end{figure}
		
		First, we need to find the length of $A$.
		\begin{flalign*}
			A &= \sqrt{H^2 + \left( L_x - L_C \right)^2} = \sqrt{P^2_z + \left( P_x - L_C \right)^2} &\\
		\end{flalign*}
		
		Using $A$ we can use the law of cosines to find $\theta_F$.
		\begin{flalign*}
			\theta_F &= \theta_\beta - \theta_\alpha &\\			
			\theta_\alpha &= \tan_1 \left( \frac{H}{L_x - L_C} \right) &\\
			L_T^2 &= L_F^2 + A^2 -2L_FA \cos (\theta_\beta) &\\ 
			\therefore \theta_F &= \cos^{-1} \left( \frac{L_T^2 - L_F^2 - A^2}{-2L_FA} \right) - \tan^{-1} \left( \frac{H}{P_x - L_C} \right) &\\
		\end{flalign*}
		
		We can also use the law of cosines to find the tibia angle as well.
		\begin{flalign*}
			\theta_T &= 90\degree - \theta_\gamma &\\			
			A^2 &= L_F^2 + L_T^2 -2L_FL_T \cos (\theta_\gamma) &\\ 
			\therefore \theta_T &= 90\degree - \cos^{-1} \left( \frac{A^2 - L_F^2 - L_T^2}{-2L_FL_T} \right) &\\
		\end{flalign*}
		
		In summary:
		\begin{flalign*}
			& \boxed{\theta_C = \tan^{-1} \left( \frac{P_y}{P_x} \right)} &\\
			& \boxed{\theta_F = \cos^{-1} \left( \frac{L_T^2 - L_F^2 - P^2_z - \left( P_x - L_C \right)^2}{-2L_F\left( \sqrt{P^2_z + \left( P_x - L_C \right)^2} \right)} \right) - \tan^{-1} \left( \frac{-P_z}{P_x - L_C} \right)} &\\
			& \boxed{\theta_T = 90\degree - \cos^{-1} \left( \frac{P^2_z + \left( P_x - L_C \right)^2 - L_F^2 - L_T^2}{-2L_FL_T} \right)} &\\
		\end{flalign*}
\end{document}